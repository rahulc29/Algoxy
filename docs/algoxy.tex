% Options for packages loaded elsewhere
\PassOptionsToPackage{unicode}{hyperref}
\PassOptionsToPackage{hyphens}{url}
%
\documentclass[
]{book}
\usepackage{amsmath,amssymb}
\usepackage{lmodern}
\usepackage{iftex}
\ifPDFTeX
  \usepackage[T1]{fontenc}
  \usepackage[utf8]{inputenc}
  \usepackage{textcomp} % provide euro and other symbols
\else % if luatex or xetex
  \usepackage{unicode-math}
  \defaultfontfeatures{Scale=MatchLowercase}
  \defaultfontfeatures[\rmfamily]{Ligatures=TeX,Scale=1}
\fi
% Use upquote if available, for straight quotes in verbatim environments
\IfFileExists{upquote.sty}{\usepackage{upquote}}{}
\IfFileExists{microtype.sty}{% use microtype if available
  \usepackage[]{microtype}
  \UseMicrotypeSet[protrusion]{basicmath} % disable protrusion for tt fonts
}{}
\makeatletter
\@ifundefined{KOMAClassName}{% if non-KOMA class
  \IfFileExists{parskip.sty}{%
    \usepackage{parskip}
  }{% else
    \setlength{\parindent}{0pt}
    \setlength{\parskip}{6pt plus 2pt minus 1pt}}
}{% if KOMA class
  \KOMAoptions{parskip=half}}
\makeatother
\usepackage{xcolor}
\IfFileExists{xurl.sty}{\usepackage{xurl}}{} % add URL line breaks if available
\IfFileExists{bookmark.sty}{\usepackage{bookmark}}{\usepackage{hyperref}}
\hypersetup{
  pdftitle={Elementary Algorithms Solutions},
  pdfauthor={Rahul Chhabra},
  hidelinks,
  pdfcreator={LaTeX via pandoc}}
\urlstyle{same} % disable monospaced font for URLs
\usepackage{color}
\usepackage{fancyvrb}
\newcommand{\VerbBar}{|}
\newcommand{\VERB}{\Verb[commandchars=\\\{\}]}
\DefineVerbatimEnvironment{Highlighting}{Verbatim}{commandchars=\\\{\}}
% Add ',fontsize=\small' for more characters per line
\usepackage{framed}
\definecolor{shadecolor}{RGB}{248,248,248}
\newenvironment{Shaded}{\begin{snugshade}}{\end{snugshade}}
\newcommand{\AlertTok}[1]{\textcolor[rgb]{0.94,0.16,0.16}{#1}}
\newcommand{\AnnotationTok}[1]{\textcolor[rgb]{0.56,0.35,0.01}{\textbf{\textit{#1}}}}
\newcommand{\AttributeTok}[1]{\textcolor[rgb]{0.77,0.63,0.00}{#1}}
\newcommand{\BaseNTok}[1]{\textcolor[rgb]{0.00,0.00,0.81}{#1}}
\newcommand{\BuiltInTok}[1]{#1}
\newcommand{\CharTok}[1]{\textcolor[rgb]{0.31,0.60,0.02}{#1}}
\newcommand{\CommentTok}[1]{\textcolor[rgb]{0.56,0.35,0.01}{\textit{#1}}}
\newcommand{\CommentVarTok}[1]{\textcolor[rgb]{0.56,0.35,0.01}{\textbf{\textit{#1}}}}
\newcommand{\ConstantTok}[1]{\textcolor[rgb]{0.00,0.00,0.00}{#1}}
\newcommand{\ControlFlowTok}[1]{\textcolor[rgb]{0.13,0.29,0.53}{\textbf{#1}}}
\newcommand{\DataTypeTok}[1]{\textcolor[rgb]{0.13,0.29,0.53}{#1}}
\newcommand{\DecValTok}[1]{\textcolor[rgb]{0.00,0.00,0.81}{#1}}
\newcommand{\DocumentationTok}[1]{\textcolor[rgb]{0.56,0.35,0.01}{\textbf{\textit{#1}}}}
\newcommand{\ErrorTok}[1]{\textcolor[rgb]{0.64,0.00,0.00}{\textbf{#1}}}
\newcommand{\ExtensionTok}[1]{#1}
\newcommand{\FloatTok}[1]{\textcolor[rgb]{0.00,0.00,0.81}{#1}}
\newcommand{\FunctionTok}[1]{\textcolor[rgb]{0.00,0.00,0.00}{#1}}
\newcommand{\ImportTok}[1]{#1}
\newcommand{\InformationTok}[1]{\textcolor[rgb]{0.56,0.35,0.01}{\textbf{\textit{#1}}}}
\newcommand{\KeywordTok}[1]{\textcolor[rgb]{0.13,0.29,0.53}{\textbf{#1}}}
\newcommand{\NormalTok}[1]{#1}
\newcommand{\OperatorTok}[1]{\textcolor[rgb]{0.81,0.36,0.00}{\textbf{#1}}}
\newcommand{\OtherTok}[1]{\textcolor[rgb]{0.56,0.35,0.01}{#1}}
\newcommand{\PreprocessorTok}[1]{\textcolor[rgb]{0.56,0.35,0.01}{\textit{#1}}}
\newcommand{\RegionMarkerTok}[1]{#1}
\newcommand{\SpecialCharTok}[1]{\textcolor[rgb]{0.00,0.00,0.00}{#1}}
\newcommand{\SpecialStringTok}[1]{\textcolor[rgb]{0.31,0.60,0.02}{#1}}
\newcommand{\StringTok}[1]{\textcolor[rgb]{0.31,0.60,0.02}{#1}}
\newcommand{\VariableTok}[1]{\textcolor[rgb]{0.00,0.00,0.00}{#1}}
\newcommand{\VerbatimStringTok}[1]{\textcolor[rgb]{0.31,0.60,0.02}{#1}}
\newcommand{\WarningTok}[1]{\textcolor[rgb]{0.56,0.35,0.01}{\textbf{\textit{#1}}}}
\usepackage{longtable,booktabs,array}
\usepackage{calc} % for calculating minipage widths
% Correct order of tables after \paragraph or \subparagraph
\usepackage{etoolbox}
\makeatletter
\patchcmd\longtable{\par}{\if@noskipsec\mbox{}\fi\par}{}{}
\makeatother
% Allow footnotes in longtable head/foot
\IfFileExists{footnotehyper.sty}{\usepackage{footnotehyper}}{\usepackage{footnote}}
\makesavenoteenv{longtable}
\usepackage{graphicx}
\makeatletter
\def\maxwidth{\ifdim\Gin@nat@width>\linewidth\linewidth\else\Gin@nat@width\fi}
\def\maxheight{\ifdim\Gin@nat@height>\textheight\textheight\else\Gin@nat@height\fi}
\makeatother
% Scale images if necessary, so that they will not overflow the page
% margins by default, and it is still possible to overwrite the defaults
% using explicit options in \includegraphics[width, height, ...]{}
\setkeys{Gin}{width=\maxwidth,height=\maxheight,keepaspectratio}
% Set default figure placement to htbp
\makeatletter
\def\fps@figure{htbp}
\makeatother
\setlength{\emergencystretch}{3em} % prevent overfull lines
\providecommand{\tightlist}{%
  \setlength{\itemsep}{0pt}\setlength{\parskip}{0pt}}
\setcounter{secnumdepth}{5}
\usepackage{booktabs}
\ifLuaTeX
  \usepackage{selnolig}  % disable illegal ligatures
\fi
\usepackage[]{natbib}
\bibliographystyle{apalike}

\title{Elementary Algorithms Solutions}
\author{Rahul Chhabra}
\date{2022-03-13}

\begin{document}
\maketitle

{
\setcounter{tocdepth}{1}
\tableofcontents
}
\hypertarget{about}{%
\chapter*{About}\label{about}}
\addcontentsline{toc}{chapter}{About}

Solutions to the text \href{https://github.com/liuxinyu95/AlgoXY}{``Elementary Algorithms''}.

For fun, non-serious learning project.

The text is particularly cool because it uses both functional and imperative techniques. I've been wanting to get better at FP for a while - so why not actually \emph{do something} in it?

Other options include using Elm for building an HTML UI (not particularly interesting if you don't already have a good idea), using F\# and Suave to build servers on ASP.NET (requires more knowledge of Kleisli categories and monads than I currently possess), or write a small computer algebra system (something I plan on doing \emph{eventually}).

Studying a few standard data structures is a particularly good option then.

Solutions are available online on \href{https://rahulc29.github.io/Algoxy/}{my Github pages}.

\hypertarget{lists}{%
\chapter{Lists}\label{lists}}

Firstly, we present a few perspectives to think about lists :

\hypertarget{high-level-declarative-perspective}{%
\section{High-level Declarative Perspective}\label{high-level-declarative-perspective}}

We'd pick a functional language for this, and this is where algebraic types come in :

\begin{Shaded}
\begin{Highlighting}[]
\KeywordTok{type}\NormalTok{ List\textless{}\textquotesingle{}a\textgreater{} = Empty | Node }\KeywordTok{of}\NormalTok{ \textquotesingle{}a * List\textless{}\textquotesingle{}a\textgreater{}}
\end{Highlighting}
\end{Shaded}

The \texttt{\textbar{}} operator represents disjoint union and \texttt{*} represents cartesian product.

\hypertarget{high-level-imperative-perspective}{%
\section{High-level Imperative Perspective}\label{high-level-imperative-perspective}}

Let's say we're in a garbage collected language with references and inheritance.

We can easily do this in a language like Kotlin :

\begin{Shaded}
\begin{Highlighting}[]
\KeywordTok{sealed} \KeywordTok{class}\NormalTok{ List}\OperatorTok{\textless{}}\DataTypeTok{T}\OperatorTok{\textgreater{}} \OperatorTok{\{}
    \KeywordTok{class}\NormalTok{ Empty}\OperatorTok{\textless{}}\DataTypeTok{T}\OperatorTok{\textgreater{}} \OperatorTok{:} \DataTypeTok{List}\OperatorTok{\textless{}}\DataTypeTok{T}\OperatorTok{\textgreater{}()}
    \KeywordTok{class}\NormalTok{ Node}\OperatorTok{\textless{}}\DataTypeTok{T}\OperatorTok{\textgreater{}(}\KeywordTok{val} \KeywordTok{data}\OperatorTok{:} \DataTypeTok{T}\OperatorTok{,} \KeywordTok{val} \VariableTok{next}\OperatorTok{:} \DataTypeTok{List}\NormalTok{\textless{}}\VariableTok{T}\NormalTok{\textgreater{}}\OperatorTok{):} \DataTypeTok{List}\OperatorTok{\textless{}}\DataTypeTok{T}\OperatorTok{\textgreater{}()}
\NormalTok{\}}
\end{Highlighting}
\end{Shaded}

This is not as pretty as the Declarative Perspective but at least there's pattern matching!

\begin{Shaded}
\begin{Highlighting}[]
\ControlFlowTok{when} \OperatorTok{(}\NormalTok{list}\OperatorTok{)} \OperatorTok{\{}
    \KeywordTok{is}\NormalTok{ List}\OperatorTok{.}\NormalTok{Node}\OperatorTok{\textless{}}\KeywordTok{Int}\OperatorTok{\textgreater{}} \OperatorTok{{-}\textgreater{}}\NormalTok{ list}\OperatorTok{.}\KeywordTok{data}
    \KeywordTok{is}\NormalTok{ List}\OperatorTok{.}\NormalTok{Empty}\OperatorTok{\textless{}*\textgreater{}} \OperatorTok{{-}\textgreater{}} \KeywordTok{null}
\OperatorTok{\}} \CommentTok{// return type of when is \textasciigrave{}Int?\textasciigrave{}}
\end{Highlighting}
\end{Shaded}

\hypertarget{low-level-imperative-perspective}{%
\section{Low-level Imperative Perspective}\label{low-level-imperative-perspective}}

Let us start by implementing a simple list imperatively, in Rust.

\begin{Shaded}
\begin{Highlighting}[]
\KeywordTok{struct}\NormalTok{ List}\OperatorTok{\textless{}}\NormalTok{T}\OperatorTok{\textgreater{}} \OperatorTok{\{}
\NormalTok{    data}\OperatorTok{:}\NormalTok{ T}\OperatorTok{,}
\NormalTok{    next}\OperatorTok{:} \DataTypeTok{Box}\OperatorTok{\textless{}}\NormalTok{List}\OperatorTok{\textless{}}\NormalTok{T}\OperatorTok{\textgreater{}\textgreater{}}
\OperatorTok{\}}
\end{Highlighting}
\end{Shaded}

This code is not particularly generic, making it for more generic requires the use of lifetimes :

\begin{Shaded}
\begin{Highlighting}[]
\KeywordTok{struct}\NormalTok{ List}\OperatorTok{\textless{}}\NormalTok{T}\OperatorTok{,} \OtherTok{\textquotesingle{}a}\OperatorTok{\textgreater{}} \OperatorTok{\{}
\NormalTok{    data}\OperatorTok{:}\NormalTok{ T}\OperatorTok{,}
\NormalTok{    next}\OperatorTok{:} \OperatorTok{\&}\OtherTok{\textquotesingle{}a}\NormalTok{ List}\OperatorTok{\textless{}}\NormalTok{T}\OperatorTok{,} \OtherTok{\textquotesingle{}a}\OperatorTok{\textgreater{}}
\OperatorTok{\}}
\end{Highlighting}
\end{Shaded}

Also, we have currently used one single lifetime \texttt{\textquotesingle{}a} for all nodes in our list.
We might want to change it later.

Do note that this \texttt{List} is immutable. This is intentional : by default, everything in Rust is immutable.

We can make it mutable as follows :

\begin{Shaded}
\begin{Highlighting}[]
\KeywordTok{struct}\NormalTok{ List}\OperatorTok{\textless{}}\NormalTok{T}\OperatorTok{,} \OtherTok{\textquotesingle{}a}\OperatorTok{\textgreater{}} \OperatorTok{\{}
\NormalTok{    data}\OperatorTok{:}\NormalTok{ T}\OperatorTok{,}
\NormalTok{    next}\OperatorTok{:} \OperatorTok{\&}\KeywordTok{mut} \OtherTok{\textquotesingle{}a}\NormalTok{ List}\OperatorTok{\textless{}}\NormalTok{T}\OperatorTok{,} \OtherTok{\textquotesingle{}a}\OperatorTok{\textgreater{}}
\OperatorTok{\}}
\end{Highlighting}
\end{Shaded}

We can also implement this type algebraically :

\begin{Shaded}
\begin{Highlighting}[]
\KeywordTok{enum}\NormalTok{ List}\OperatorTok{\textless{}}\NormalTok{T}\OperatorTok{,} \OtherTok{\textquotesingle{}a}\OperatorTok{\textgreater{}} \OperatorTok{\{}
\NormalTok{    Empty}\OperatorTok{,}
\NormalTok{    Node }\OperatorTok{\{}
\NormalTok{        data}\OperatorTok{:}\NormalTok{ T}\OperatorTok{,}
\NormalTok{        next}\OperatorTok{:} \OperatorTok{\&}\OtherTok{\textquotesingle{}a}\NormalTok{ List}\OperatorTok{\textless{}}\NormalTok{T}\OperatorTok{,} \OtherTok{\textquotesingle{}a}\OperatorTok{\textgreater{}}
    \OperatorTok{\}}
\OperatorTok{\}}
\end{Highlighting}
\end{Shaded}

Also, we are currently constrained in using only one kind of reference (native \texttt{\&} and \texttt{\&mut}). In reality, we would want our \texttt{List} to be \emph{polymorphic} over all possible kinds of references.

We can do this with Rust's \texttt{Deref} trait.

\begin{Shaded}
\begin{Highlighting}[]
\KeywordTok{enum}\NormalTok{ List}\OperatorTok{\textless{}}\NormalTok{T}\OperatorTok{\textgreater{}} \OperatorTok{\{}
\NormalTok{    Empty}\OperatorTok{,}
\NormalTok{    Node }\OperatorTok{\{}
\NormalTok{        data}\OperatorTok{:}\NormalTok{ T}\OperatorTok{,}
\NormalTok{        next}\OperatorTok{:} \KeywordTok{dyn} \BuiltInTok{Deref}\OperatorTok{\textless{}}\NormalTok{Target }\OperatorTok{=}\NormalTok{ List}\OperatorTok{\textless{}}\NormalTok{T}\OperatorTok{\textgreater{}\textgreater{}}
    \OperatorTok{\}}
\OperatorTok{\}}
\end{Highlighting}
\end{Shaded}

The low-level implementation certainly requires a lot more work due to the lack of the garbage collector!

\hypertarget{exercise-1.2}{%
\section{Exercise 1.2}\label{exercise-1.2}}

\begin{quote}
For list of type \(A\), suppose we can test if any two elements \(x\), \(y\) \(\in\) \(A\) are equal,
define an algorithm to test if two lists are identical.
\end{quote}

First we define a recursive algorithm with HLFP :

\begin{Shaded}
\begin{Highlighting}[]
\KeywordTok{let} \KeywordTok{rec}\NormalTok{ Equals a b elementEquals = }
    \KeywordTok{match}\NormalTok{ a }\KeywordTok{with} 
\NormalTok{    | }\OperatorTok{[]}\NormalTok{ {-}\textgreater{} }\KeywordTok{match}\NormalTok{ b }\KeywordTok{with} 
\NormalTok{        | }\OperatorTok{[]}\NormalTok{ {-}\textgreater{} }\KeywordTok{true} 
\NormalTok{        | head::tail {-}\textgreater{} }\KeywordTok{false} 
\NormalTok{    | aHead::aTail {-}\textgreater{} }\KeywordTok{match}\NormalTok{ b }\KeywordTok{with} 
\NormalTok{        | }\OperatorTok{[]}\NormalTok{ {-}\textgreater{} }\KeywordTok{false} 
\NormalTok{        | bHead::bTail {-}\textgreater{} }\OperatorTok{(}\NormalTok{elementEquals aHead bHead}\OperatorTok{)}\NormalTok{ \&\& }\OperatorTok{(}\NormalTok{Equals aTail bTail elementEquals}\OperatorTok{)}
\end{Highlighting}
\end{Shaded}

This algorithm is elegant and readable but suffers from consuming \(O(n)\) stack space.

Let us make it \emph{tail-recursive}!

\begin{Shaded}
\begin{Highlighting}[]
\KeywordTok{let}\NormalTok{ Equals a b elementEquals = }
    \KeywordTok{let} \KeywordTok{rec}\NormalTok{ Loop a b result = }
        \KeywordTok{if}\NormalTok{ !result }\KeywordTok{then}\NormalTok{ result }\KeywordTok{else} 
            \KeywordTok{match}\NormalTok{ a }\KeywordTok{with} 
\NormalTok{            | }\OperatorTok{[]}\NormalTok{ {-}\textgreater{} }\KeywordTok{match}\NormalTok{ b }\KeywordTok{with} 
\NormalTok{                | }\OperatorTok{[]}\NormalTok{ {-}\textgreater{} }\KeywordTok{true} 
\NormalTok{                | head::tail {-}\textgreater{} }\KeywordTok{false} 
\NormalTok{            | aHead::aTail {-}\textgreater{} }\KeywordTok{match}\NormalTok{ b }\KeywordTok{with} 
\NormalTok{                | }\OperatorTok{[]}\NormalTok{ {-}\textgreater{} }\KeywordTok{false} 
\NormalTok{                | bHead::bTail {-}\textgreater{} Loop aTail bTail }\OperatorTok{(}\NormalTok{result \&\& }\OperatorTok{(}\NormalTok{elementEquals aHead bHead}\OperatorTok{))}
\NormalTok{    Loop a b }\KeywordTok{true}
\end{Highlighting}
\end{Shaded}


  \bibliography{book.bib,packages.bib}

\end{document}
